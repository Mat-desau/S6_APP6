% !TEX encoding = UTF-8 Unicode
% !TEX TS-program = pdflatex
\documentclass[DIV=15,paper=letter,titlepage=true,fontsize=12pt,headings=normal,captions=nooneline]{scrartcl}

%% Configuration de base pour l'encodage et pour la langue:
%% version du 2014/06/21 pour babel-french
%% \usepackage{francais-utf8}

% Encodage caractère et fonte
\usepackage[utf8]{inputenc}
\usepackage[T1]{fontenc}

% Langue: français	
\usepackage[frenchb]{babel}
\frenchbsetup{og=«,fg=»}
\usepackage{xspace}

% Utilisation de \nombre
\usepackage[autolanguage]{numprint} 
%% Fin pour encodage et langue

\usepackage[automark,markcase=upper]{scrlayer-scrpage}
\usepackage[protrusion=true,expansion]{microtype}
\usepackage{amsmath}
\usepackage{graphicx}
\usepackage{array,booktabs,longtable,tabularx}
\usepackage{arydshln}
\usepackage{dashrule}
\usepackage[squaren,cdot,textstyle]{SIunits}
\usepackage{ccicons}
\usepackage{enumerate}
\usepackage{url}

\usepackage{xcolor}
\definecolor{myred}{rgb}{.7,.13,.13}
\definecolor{myyellow}{rgb}{1,1,.79}
\definecolor{monbleu}{rgb}{.2,.4,.8}

\definecolor{vertUdS}{RGB}{0,119,73} % Pantone vert 3415
\definecolor{bleuTetrad}{RGB}{0,45,117} % Pantone Tetrad
\definecolor{rougTetrad}{RGB}{117,0,45} % Pantone Tetrad
\definecolor{oranTetrad}{RGB}{117,72,0} % Pantone Tetrad

% Sérif: lining figures in math, osf in text
\usepackage[scaled=.98,sups]{XCharter}% 
% Sans sérif:
\usepackage[scaled=0.90]{helvet}%
% Mono:
\usepackage[scaled=0.85]{beramono}
% Math:
\usepackage[charter,bigdelims,vvarbb,scaled=1.07]{newtxmath}
\usepackage[cal=boondoxo]{mathalfa} % doc à regarder: intéressant.

% Math: autre \usepackage[osf,sc]{mathpazo}

\linespread{1.1}

% À voir:
%\DeclareFixedFont{\FonteTitre}{T1}{phv}{b}{n}{48pt}
\usepackage{pifont}
\usepackage[os=mac]{menukeys}

%% Dimension de la page:
%\setlength{\textheight}{672pt}
%\setlength{\footskip}{24pt}

% Réglage de certaine apparence:
% Caption
\setkomafont{caption}{\sffamily}
\setkomafont{captionlabel}{\bfseries}
\setcapindent{0em}
% titre: section et autre
\addtokomafont{section}{%
	\color{rougTetrad}%
	\usesizeofkomafont{subsection}%
	}%
% Entête et pied de page:
\ihead{}
\chead{}
\ohead{\sffamily\scriptsize\scshape\textls[150]{\headmark}}
\ifoot{}
\cfoot{}
\ofoot{\normalfont\sffamily\small\thepage}
\renewcommand*{\sectionmarkformat}{}% pas de numéro de section

% Pas d'indentation
\setlength{\parindent}{0pt}

\begin{document}
%%% Titre
\thispagestyle{empty}
\noindent
\begin{center}
{\bfseries\sffamily \textls[150]{Université de Sherbrooke}\\[4pt] \textls[75]{Département de génie électrique et de génie informatique}}

\vspace{3cm}
\textsf{\large Session S6e 2025}

\vspace{.5cm}
\textsf{\Large Rapport de l'unité 6}

\vspace{.25cm}
\textsf{\footnotesize remis le jour mois 2024}

\end{center}
\vfill

\noindent
\rule{\linewidth}{.8pt}

\noindent
Nom 1, Prénom 1 \hfill CIP 1\\
Nom 2, Prénom 2 \hfill CIP 2\\

\newpage
\section{Introduction}
Le rapport propose une démarche afin de vous guider dans la problématique

\section{Ligne principale: 20 points}
%Ok, le contexte n'est pas clair: en 2024: précision pour la section 2.1: dans la question, on cherche à reproduire de façon manuelle soit la figure 1 soit la figure 2 du guide étudiant (à votre choix). Vous pouvez ne pas aller jusqu'à 20ns si vous justifiez le pourquoi.

On considère un circuit (voir figure 1a de la problématique) avec la même source, la même résistance de source. La ligne de transmission utilisé est \menu{TLIND} dans la librairie \menu{Tlines-Ideal}. On choisira l'impédance caractéristique à \unit{50}{\ohm} et un temps de propagation à \unit{1}{\nano\sec}. La charge capacitive est remplacée par un circuit ouvert, soit une résistance de forte valeur ($R2$ sur la figure).
\begin{enumerate}
	\item Faites les démarches et calculs nécessaires pour tracer les tensions $V_1(t)$ et $V_2(t)$ pour $0\leq t\leq \unit{20}{\nano\sec}$ données à la Figure~1a du guide étudiant. Laisser les traces de vos calculs et valider avec la simulation donnée à la Figure~1b.
	\item Par simulation, changer le temps tRise tel que tRiseb=tFall et $\textrm{tWidth (ns)} = 5 - \textrm{tRise} - \textrm{tFall}$. Présenter vos résultat sur une figure. Est-ce une piste de solution pour résoudre le problème sur la ligne principale?
	\item Par simulation, changer le délai sur la ligne. Présenter vos résultat sur une figure. Est-ce une piste de solution pour résoudre le problème  sur la ligne principale?
\end{enumerate}

\paragraph{Barème} 
\begin{enumerate}
	\item représentation théorique des signaux V1 et V2: 10 points
	\item Simulation avec variations du temps de montée et descente du signal numérique et explications: 5 points
	\item Simulations avec variations du temps de propagation et explications:  5 points
\end{enumerate}


\section{Lignes principale et secondaire: 20 points}
On utilise le circuit de la figure~4a de la problématique avec la même source, la même résistance de source. La charge capacitive est remplacée par un circuit ouvert, soit une résistance de forte valeur ($R2$ sur la figure).

La ligne de transmission utilisé est \menu{CLIN} dans la librairie \menu{Tlines-Ideal}.

\begin{enumerate} 
	\item Déterminer les dimensions de la ligne pour obtenir une impédance caractéristique de \unit{50}{\ohm} (ne pas oublier qu'il y a deux rubans).
	\item Si on adapte en entrée ou en sortie (mettre le résultat de vos simulations), quelles tensions sont améliorées?
	\item Peut-on améliorer les autres tensions en jouant sur les temps de montée et de descente ? Donner votre meilleur résultat de simulation et expliquer les problèmes résiduels.
\end{enumerate}

\paragraph{Points} 
\begin{enumerate}
	\item Résultats des valeurs obtenues pour avoir une ligne à \unit{50}{\ohm}: 5 points
	\item Adaptation en entrée et en sortie, meilleur choix et explications: 10 points
	\item Amélioration des autres tensions: méthodes et explications: 5 points
\end{enumerate}

\section{Circuit final: 10 points}
Trouver une méthode permettant d'améliorer significativement le problème restant et montrer le résultat par simulation.

\paragraph{Points} 
\begin{enumerate}
	\item Méthode proposée: 2 points
	\item Démonstration par simulation que cela semble fonctionner: 8 points
\end{enumerate}

\section{Conclusion: 10 points}
Donner les pistes de solutions pour résoudre la problématique pour le client. Selon votre avis, faudra-t-il refaire le PCB ?

\paragraph{Points} 
\begin{enumerate}
	\item Pistes de solution: 5 points
	\item Justification pour refaire le PCB (rester dans le cadre de la problématique): 5 points
\end{enumerate}

\end{document}